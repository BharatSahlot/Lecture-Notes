\section{Lecture 5}

\subsection{Terms}

\paragraph{ARS} Abstract Reduction System

\paragraph{Terminating ARS} An ARS is terminating \emph{iff.} it has no infinite runs.

\subsection{Well Founded Induction(WFI)}

This is property of an abstract reduction system. A system having this property implies that,

$$
\forall x \in A \ \left[ \forall y \in A, x \successor y \implies P(y) \right] \implies P(x)
$$

in other words,
$P(x)$ is satisfied if $\forall y, x \successor y$, $P(y)$ is satisfied.

\begin{theorem}
    Let, $(A, \longrightarrow)$ be an ARS, then $A$ satisfies WFI iff. $(A, \longrightarrow)$ is terminating.
\end{theorem}

\begin{proof}
    \textbf{TODO}
    \text{} \\
    \textbf{1. $(A, \longrightarrow)$ terminates $\implies A$ satisfies WFI.} \\
    \\
    \textbf{2. $A$ satisfies WFI $\implies (A, \longrightarrow)$ terminates.}
\end{proof}

\subsection{Confluence}

Order of evaluation does not matter.

\paragraph{Joinable} $x$ and $y$ are joinable, denoted by $\downarrow$ \emph{iff.} they have the same normal form.

\paragraph{Local Confluence} An element $x \in A$ is said to be locally confluent if $\forall y, z \in A, x \successor y, z$ $\exists w : y \too w, z \too w$, in other words, $y \downarrow z$.

\paragraph{\rightarrow} If a system is terminating and has local confluence for all vertices, then the system has confluence.

\subsection{Interative Evaluation}

\textbf{TODO}
